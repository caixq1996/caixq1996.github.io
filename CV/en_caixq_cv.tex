%%%%%%%%%%%%%%%%%%%%%%%%%%%%%%%%%%%%%%%%%%%%%%%%%%%%%%%%%%%%%%%%%%%%%%%%
%%%%%%%%%%%%%%%%%%%%%% Simple LaTeX CV Template %%%%%%%%%%%%%%%%%%%%%%%%
%%%%%%%%%%%%%%%%%%%%%%%%%%%%%%%%%%%%%%%%%%%%%%%%%%%%%%%%%%%%%%%%%%%%%%%%

%%%%%%%%%%%%%%%%%%%%%%%%%%%%%%%%%%%%%%%%%%%%%%%%%%%%%%%%%%%%%%%%%%%%%%%%
%% NOTE: If you find that it says                                     %%
%%                                                                    %%
%%                           1 of ??                                  %%
%%                                                                    %%
%% at the bottom of your first page, this means that the AUX file     %%
%% was not available when you ran LaTeX on this source. Simply RERUN  %%
%% LaTeX to get the ``??'' replaced with the number of the last page  %%
%% of the document. The AUX file will be generated on the first run   %%
%% of LaTeX and used on the second run to fill in all of the          %%
%% references.                                                        %%
%%%%%%%%%%%%%%%%%%%%%%%%%%%%%%%%%%%%%%%%%%%%%%%%%%%%%%%%%%%%%%%%%%%%%%%%

%%%%%%%%%%%%%%%%%%%%%%%%%%%% Document Setup %%%%%%%%%%%%%%%%%%%%%%%%%%%%

% Don't like 10pt? Try 11pt or 12pt
\documentclass[10pt]{article}

% This is a helpful package that puts math inside length specifications
\usepackage{calc}
\usepackage{amssymb}
\usepackage{amsmath}
\usepackage{CTEX}
% \documentclass[10pt, UTF8]{ctexart}


% Simpler bibsection for CV sections
% (thanks to natbib for inspiration)



% Layout: Puts the section titles on left side of page
\reversemarginpar

%
%         PAPER SIZE, PAGE NUMBER, AND DOCUMENT LAYOUT NOTES:
%
% The next \usepackage line changes the layout for CV style section
% headings as marginal notes. It also sets up the paper size as either
% letter or A4. By default, letter was used. If A4 paper is desired,
% comment out the letterpaper lines and uncomment the a4paper lines.
%
% As you can see, the margin widths and section title widths can be
% easily adjusted.
%
% ALSO: Notice that the includefoot option can be commented OUT in order
% to put the PAGE NUMBER *IN* the bottom margin. This will make the
% effective text area larger.
%
% IF YOU WISH TO REMOVE THE ``of LASTPAGE'' next to each page number,
% see the note about the +LP and -LP lines below. Comment out the +LP
% and uncomment the -LP.
%
% IF YOU WISH TO REMOVE PAGE NUMBERS, be sure that the includefoot line
% is uncommented and ALSO uncomment the \pagestyle{empty} a few lines
% below.
%

% Use these lines for letter-sized paper
\usepackage[paper=letterpaper,
            %includefoot, % Uncomment to put page number above margin
            marginparwidth=1.15in,     % Length of section titles
            marginparsep=.05in,       % Space between titles and text
            margin=0.7in,               % 1 inch margins
            includemp]{geometry}

%% Use these lines for A4-sized paper
%\usepackage[paper=a4paper,
%            %includefoot, % Uncomment to put page number above margin
%            marginparwidth=30.5mm,    % Length of section titles
%            marginparsep=1.5mm,       % Space between titles and text
%            margin=25mm,              % 25mm margins
%            includemp]{geometry}

%% More layout: Get rid of indenting throughout entire document
\setlength{\parindent}{0in}
\newcommand\tab[1][1cm]{\hspace*{#1}}

%% This gives us fun enumeration environments. compactitem will be nice.
\usepackage{paralist}
%% Reference the last page in the page number
%
% NOTE: comment the +LP line and uncomment the -LP line to have page
%       numbers without the ``of ##'' last page reference)
%
% NOTE: uncomment the \pagestyle{empty} line to get rid of all page
%       numbers (make sure includefoot is commented out above)
%
\usepackage{fancyhdr,lastpage}
\pagestyle{fancy}
%\pagestyle{empty}      % Uncomment this to get rid of page numbers
\fancyhf{}\renewcommand{\headrulewidth}{0pt}
\fancyfootoffset{\marginparsep+\marginparwidth}
\newlength{\footpageshift}
\setlength{\footpageshift}
          {0.5\textwidth+0.5\marginparsep+0.5\marginparwidth-2in}
\lfoot{\hspace{\footpageshift}%
       \parbox{4in}{\, \hfill %
                    \arabic{page} of \protect\pageref*{LastPage} % +LP
%                    \arabic{page}                               % -LP
                    \hfill \,}}

% Finally, give us PDF bookmarks
\usepackage{color}
% \usepackage[hypertex,dvipdfmx]{hyperref}
\usepackage{hyperref}
\definecolor{darkblue}{rgb}{0.0,0.0,0.3}
\hypersetup{colorlinks,breaklinks,
            linkcolor=darkblue,urlcolor=darkblue,
            anchorcolor=darkblue,citecolor=darkblue}

%%%%%%%%%%%%%%%%%%%%%%%% End Document Setup %%%%%%%%%%%%%%%%%%%%%%%%%%%%


%%%%%%%%%%%%%%%%%%%%%%%%%%% Helper Commands %%%%%%%%%%%%%%%%%%%%%%%%%%%%

% The title (name) with a horizontal rule under it
%
% Usage: \makeheading{name}
%
% Place at top of document. It should be the first thing.
%\newcommand{\makeheading}[1]%
%        {\hspace*{-\marginparsep minus \marginparwidth}%
%         \begin{minipage}[t]{\textwidth+\marginparwidth+\marginparsep}%
%                {\large \bfseries #1}\\[-0.15\baselineskip]%
%                 \rule{\columnwidth}{1pt}%
%         \end{minipage}}

\newcommand{\makeheading}[2][]%
        {\hspace*{-\marginparsep minus \marginparwidth}%
         \begin{minipage}[t]{\textwidth+\marginparwidth+\marginparsep}%
             {\large \bfseries #2 \hfill #1}\\[-0.15\baselineskip]%
                 \rule{\columnwidth}{1pt}%
         \end{minipage}}
% The section headings
%
% Usage: \section{section name}
%
% Follow this section IMMEDIATELY with the first line of the section
% text. Do not put whitespace in between. That is, do this:
%
%       \section{My Information}
%       Here is my information.
%
% and NOT this:
%
%       \section{My Information}
%
%       Here is my information.
%
% Otherwise the top of the section header will not line up with the top
% of the section. Of course, using a single comment character (%) on
% empty lines allows for the function of the first example with the
% readability of the second example.
\renewcommand{\section}[2]%
        {\pagebreak[2]\vspace{1.3\baselineskip}%
         \phantomsection\addcontentsline{toc}{section}{#1}%
         \hspace{0in}%
         \marginpar{
         \raggedright \scshape #1}#2}

% An itemize-style list with lots of space between items
\newenvironment{outerlist}[1][\enskip\textbullet]%
        {\begin{itemize}[#1]}{\end{itemize}%
         \vspace{-.6\baselineskip}}

% An environment IDENTICAL to outerlist that has better pre-list spacing
% when used as the first thing in a \section
\newenvironment{lonelist}[1][\enskip\textbullet]%
        {\vspace{-\baselineskip}\begin{list}{#1}{%
        \setlength{\partopsep}{0pt}%
        \setlength{\topsep}{0pt}}}
        {\end{list}\vspace{-.6\baselineskip}}

% An itemize-style list with little space between items
\newenvironment{innerlist}[1][\enskip\textbullet]%
        {\begin{compactitem}[#1]}{\end{compactitem}}

% An environment IDENTICAL to innerlist that has better pre-list spacing
% when used as the first thing in a \section
\newenvironment{loneinnerlist}[1][\enskip\textbullet]%
        {\vspace{-\baselineskip}\begin{compactitem}[#1]}
        {\end{compactitem}\vspace{-.6\baselineskip}}



% To add some paragraph space between lines.
% This also tells LaTeX to preferably break a page on one of these gaps
% if there is a needed pagebreak nearby.
\newcommand{\blankline}{\quad\pagebreak[2]}
\newcommand{\halfblankline}{\quad\vspace{-0.5\baselineskip}\pagebreak[3]}

% Uses hyperref to link DOI
\newcommand\doilink[1]{\href{http://dx.doi.org/#1}{#1}}
\newcommand\doi[1]{doi:\doilink{#1}}

\usepackage[shortlabels]{enumitem}

\makeatletter

\newcounter{Lcount}
\newlength{\bibhang}
\setlength{\bibhang}{1em}
\newlength{\bibsep}
 {\@listi \global\bibsep\itemsep \global\advance\bibsep by\parsep}
\newenvironment{bibsectiona}
    {\minipage[t]{\linewidth}\list{\arabic{Lcount}.}{%
    \usecounter{Lcount}%
    \setlength{\leftmargin}{\bibhang}%
        \setlength{\itemindent}{-\leftmargin}%
        \setlength{\itemsep}{\bibsep}%
        \setlength{\parsep}{\z@}%
        }}
    {\endlist\endminipage}

\newenvironment{bibsectionb}
    {\minipage[t]{\linewidth}\list{}{%
        \setlength{\leftmargin}{\bibhang}%
        \setlength{\itemindent}{-\leftmargin}%
        \setlength{\itemsep}{\bibsep}%
        \setlength{\parsep}{\z@}%
        }}
    {\endlist\endminipage}


\newenvironment{bibsectionc}%
        {\begin{enumerate}{}{%
%        {\begin{list}{}{%
       \setlength{\leftmargin}{\bibhang}%
       \setlength{\itemindent}{-\leftmargin}%
       \setlength{\itemsep}{\bibsep}%
       \setlength{\parsep}{\z@}%
        \setlength{\partopsep}{0pt}%
        \setlength{\topsep}{0pt}}}
        {\end{enumerate}\vspace{-.6\baselineskip}}
%        {\end{list}\vspace{-.6\baselineskip}}

\newenvironment{bibsectiond}%
        {\begin{enumerate}[labelindent=0pt,itemindent=0em,leftmargin=\bibhang,label={[\arabic*]}]{}{%
       \setlength{\itemsep}{\bibsep}%
       \setlength{\parsep}{\z@}%
        \setlength{\partopsep}{0pt}%
        \setlength{\topsep}{0pt}}}
        {\end{enumerate}\vspace{-.6\baselineskip}}


\makeatother



%%%%%%%%%%%%%%%%%%%%%%%% End Helper Commands %%%%%%%%%%%%%%%%%%%%%%%%%%%

%%%%%%%%%%%%%%%%%%%%%%%%% Begin CV Document %%%%%%%%%%%%%%%%%%%%%%%%%%%%

\begin{document}
\makeheading[Curriculum Vitae]{Xin-Qiang Cai [\href{https://caixq1996.github.io/}{Homepage}]}

\section{Contact Information}
%
% NOTE: Mind where the & separators and \\ breaks are in the following
%       table.
%
% ALSO: \rcollength is the width of the right column of the table
%       (adjust it to your liking; default is 1.85in).
%
\newlength{\rcollength}\setlength{\rcollength}{2.3in}%
%
\begin{tabular}[t]{@{}p{\textwidth-\rcollength}p{\rcollength}}
{Phone:} +81-08063217094, +86-15651682506 \\
{E-mail:} \href{cai@ms.k.u-tokyo.ac.jp}{cai@ms.k.u-tokyo.ac.jp}, \href{jkrsndivide@gmail.com}{jkrsndivide@gmail.com}\\
% Nanjing 210023, China &  \href{http://www.lamda.nju.edu.cn/caixq/}{www.lamda.nju.edu.cn/caixq/}
\end{tabular}

%\section{Research Interests}
%
%Machine Learning, Optimization

\section{Education}
\href{https://www.u-tokyo.ac.jp/en/}{\textbf{The University of Tokyo}},
Tokyo, Japan
\begin{outerlist}
\item[] Ph.D., Complexity Science and Engineering,  Oct., 2021  -- Sep., 2024 (Expected)
 \begin{innerlist}
        \item Laboratory: \href{http://www.ms.k.u-tokyo.ac.jp/index.html}{Sugiyama-Yokoya-Ishida Lab}, led by Professor \href{http://www.ms.k.u-tokyo.ac.jp/sugi/}{Masashi Sugiyama}
        \item Advisor: Professor~\href{http://www.ms.k.u-tokyo.ac.jp/sugi/}{Masashi Sugiyama}
        \item Research Area: Reinforcement Learning and Imitation Learning from Weak Supervision\\
\end{innerlist}
\end{outerlist}

%
\href{https://www.nju.edu.cn/EN/}{\textbf{Nanjing University}},
Nanjing, China
\begin{outerlist}
\item[] M.S., Computer Science and Technology,  Sep., 2018  -- Jun., 2021
 \begin{innerlist}
        \item Laboratory: \href{http://www.lamda.nju.edu.cn/MainPage.ashx}{LAMDA Lab}, led by Professor \href{http://cs.nju.edu.cn/zhouzh}{Zhi-Hua Zhou}
        \item Advisor: Co-supervised by Professor~\href{https://cs.nju.edu.cn/zhouzh/}{Zhi-Hua Zhou} and Professor~\href{https://lamda.nju.edu.cn/jiangy/}{Yuan Jiang}
        \item Research Area: Data Mining\\
\end{innerlist}
\end{outerlist}


\href{https://en.nwpu.edu.cn/}{\textbf{Northwestern Polytechnical University}}, Xi'an, China
\begin{outerlist}
\item[] B.E., Aircraft Design and Engineering, Sep., 2014  -- Jun., 2018
\end{outerlist}
\section{Research Interests}
I am interested in various topics on machine learning and data mining. My research primarily focuses on reinforcement learning and imitation learning from weak supervision.

% \section{Research \& Projects}
% %
% % {\textbf{High-Dimensional Imitation Learning}} \hfill \textbf{Jun., 2019 -- Present}
% % \begin{outerlist}
% % \item[] Master Thesis/Project supervised by Prof. Zhi-Hua Zhou and Prof. Yuan Jiang
% %  \begin{innerlist}
% %         \item Pointed out the key for adversary-based imitation learning methods with high-dimensional inputs is balancing the discrimination-rewarding trade-off
% %         \item Presented a theorem to disclose how to deal with the trade-off
% %         \item Proposed a new algorithm named HashReward to solve high-dimensional imitation learning problems by using supervised hashing, and HashReward gains state-of-the-art performance in many pixel-level environments
% %         \item The work has been submitted to AAMAS 2021
% % \end{innerlist}
% % \end{outerlist}
% {\textbf{Imitation Learning with Heterogeneous Observations}} \hfill \textbf{Jun., 2020 -- Feb., 2021}
% \begin{outerlist}
% \item[] MSc thesis research
% \begin{innerlist}
% \item Project: Machine Learning in Open and Dynamic Environments
% \item Analyze the imitation learning tasks in open environments, where the observations in the demonstrations and the environment are heterogeneous. Proposing a new algorithm to handle this task. This work is publicly available on arxiv.
% \end{innerlist}
% \end{outerlist}

% \blankline

% {\textbf{Learning from Human Based Objectives}} \hfill \textbf{Aug., 2019 -- Present}
% \begin{outerlist}
% \item[] MSc thesis research
% \begin{innerlist}
% \item Project: Human in The Loop Learning
% \item Proposing a student-teacher framework for learning under human based objective, which is hard to optimize for the traditional learning process.
% \end{innerlist}
% \end{outerlist}

% \blankline

% {\textbf{High-Dimensional Imitation Learning}} \hfill \textbf{Jun., 2019 -- Jun., 2020}
% \begin{outerlist}
% \item[] MSc thesis research
% \begin{innerlist}
% \item Project: Machine Learning in Open and Dynamic Environments
% \item Pointing out the key for adversary-based imitation learning methods with high-dimensional inputs in theory, and propose a new state-of-the-art algorithm based on supervised hashing to solve imitation learning problems in pixel-level control tasks. This work has been published in AAMAS 2021.
% \end{innerlist}
% \end{outerlist}

% \blankline

% % {\textbf{Streaming Classification with Emerging New Classes}} \hfill \textbf{Sep., 2018 -- Aug. 2019}
% % \begin{outerlist}
% % \item[] Master Thesis/Project supervised by Prof. Yuan Jiang and advised by Prof. Kai-Ming Ting
% %  \begin{innerlist}
% %         \item Identified real-world systems in streaming classification with emerging new classes (SENC) problems could contain emerging new classes both geometrically far from and near by known classes
% %         \item Refined the SENC problem into an $\alpha$-SENC problem by using a class separation indicator $\alpha$, where $\alpha$ is used to specify the degrees of difficulties of the $\alpha$ SENC problem in a single data stream
% %         \item Proposed a new algorithm named SENNE, which is based on nearest neighbor ensembles, to solve both low-$\alpha$ and high-$\alpha$ problems
% %         \item The work has been published in ICDM 2019
% % \end{innerlist}
% % \end{outerlist}

% {\textbf{Streaming Classification with Emerging New Classes}} \hfill \textbf{Sep., 2018 -- Aug. 2019}
% \begin{outerlist}
% \item[] MSc thesis research
% \begin{innerlist}
% \item Project: Machine Learning in Open and Dynamic Environments
% \item Refine the current streaming classification with emerging new classes (SENC) problem into an $\alpha$-SENC problem by using a class separation indicator $\alpha$, and propose a new algorithm based on nearest neighbor ensembles to deal with both low and high $\alpha$-SENC problem. This work has been published in ICDM 2019.
% \end{innerlist}
% \end{outerlist}

\section{Publications} \vspace{-4.2ex}
\begin{bibsectiond}%
\item  \textbf{Xin-Qiang Cai}, Pushi Zhang, Li Zhao, Jiang Bian, \href{http://www.ms.k.u-tokyo.ac.jp/sugi/}{Masashi Sugiyama}, Ashley Juan Llorens. \href{https://neurips.cc/virtual/2023/poster/70393}{Distributional Pareto-Optimal Multi-Objective Reinforcement Learning.} In: \textbf{Proceedings of the Thirty-seventh Conference on Neural Information Processing Systems (\href{https://openreview.net/group?id=NeurIPS.cc/2023/Conference}{NeurIPS'23})}, New Orleans, US, 2019. To appear.

\item  \textbf{Xin-Qiang Cai}, Yu-Jie Zhang, Chao-Kai Chiang, \href{http://www.ms.k.u-tokyo.ac.jp/sugi/}{Masashi Sugiyama}. \href{https://neurips.cc/virtual/2023/poster/72411}{Imitation Learning from Vague Feedback.} In: \textbf{Proceedings of the Thirty-seventh Conference on Neural Information Processing Systems (\href{https://openreview.net/group?id=NeurIPS.cc/2023/Conference}{NeurIPS'23})}, New Orleans, US, 2019. To appear.

\item  \textbf{Xin-Qiang Cai}, Yao-Xiang Ding, Zi-Xuan Chen, \href{http://www.lamda.nju.edu.cn/jiangy/}{Yuan Jiang}, \href{http://www.ms.k.u-tokyo.ac.jp/sugi/}{Masashi Sugiyama}, and \href{http://cs.nju.edu.cn/zhouzh/}{Zhi-Hua Zhou}. \href{https://openreview.net/forum?id=3ULaIHxn9u7}{Seeing Differently, Acting Similarly: Imitation Learning with Heterogeneous Observations.} In: \textbf{Proceedings of the Eleventh International Conference on Learning Representations (\href{http://icdm2019.bigke.org/}{ICLR'23}, spotlight)}, Kigali, Rwanda, 2023.

\item Zi-Xuan Chen*, \textbf{Xin-Qiang Cai*}, \href{http://www.lamda.nju.edu.cn/jiangy/}{Yuan Jiang}, and \href{http://cs.nju.edu.cn/zhouzh/}{Zhi-Hua Zhou}. \href{https://dl.acm.org/doi/abs/10.5555/3535850.3535879}{Anomaly Guided Policy Learning from Imperfect Demonstrations.} In: \textbf{Proceedings of the 21th International Conference on Autonomous Agents and Multi-Agent Systems (\href{https://aamas2022-conference.auckland.ac.nz/}{AAMAS'22}, oral)}, Auckland, New Zealand, 2022.

\item \textbf{Xin-Qiang Cai}, Yao-Xiang Ding, \href{http://www.lamda.nju.edu.cn/jiangy/}{Yuan Jiang}, and \href{http://cs.nju.edu.cn/zhouzh/}{Zhi-Hua Zhou}. \href{https://dl.acm.org/doi/10.5555/3463952.3463990}{Imitation Learning from Pixel-Level Demonstrations by HashReward}. In: \textbf{Proceedings of the 20th International Conference on Autonomous Agents and Multi-Agent Systems (\href{https://aamas2021.soton.ac.uk/}{AAMAS'21})}, online, 2021.

\item  \textbf{Xin-Qiang Cai}, Peng Zhao, \href{http://ai.nju.edu.cn/KaiMingTing/}{Kai Ming Ting}, Xin Mu, \href{https://lamda.nju.edu.cn/jiangy/}{Yuan Jiang}. \href{https://ieeexplore.ieee.org/document/8970887}{Nearest Neighbor Ensembles: An Effective Method for Difficult Problems in Streaming Classification with Emerging New Classes.} In: \textbf{Proceedings of the 19th IEEE International Conference on Data Mining (\href{http://icdm2019.bigke.org/}{ICDM'19})}, Beijing, China, 2019. 
\end{bibsectiond}

\section{Preprints}\vspace{-4.2ex}
\begin{bibsectiond}
% \addtocounter{enumi}{2}%
\item Kaiyan Zhao, Qiyu Wu, {\bf Xin-Qiang Cai}, Yoshimasa Tsuruoka. Leveraging Multi-lingual Positive Instances in Contrastive Learning to Improve Sentence Embedding. \href{https://arxiv.org/abs/2309.08929}{arXiv}.
\end{bibsectiond}

% \section{Patent}\vspace{-4.2ex}
% 一种摄像器材记录的视频图像数据的高维模仿学习方法. Patent No. 202011450396.1. 2020.

\section{Honors and Awards}
\textbf{Contest Awards}
\begin{outerlist}
\item[]
\begin{innerlist}
\item \makebox[2.6cm][l]{2018} First Prize, ZhongAn Hackathon Contest, AI scenario
\item \makebox[2.6cm][l]{2017} Sliver Medal, ACM/ICPC Asia Regional Contest, Qingdao Site
\item \makebox[2.6cm][l]{2017} Bronze Medal, ACM/ICPC Asia Regional Contest, Xinjiang Site
\item \makebox[2.6cm][l]{2016, 2017} First Prize, China Collegiate Programming Contest
\item \makebox[2.6cm][l]{2016, 2017} Honorable Mention, Interdisciplinary Contest In Modeling
\item \makebox[2.6cm][l]{2016} Bronze Medal, ACM/ICPC Asia Regional Contest, Dalian Site
\item \makebox[2.6cm][l]{2016} Bronze Medal, ACM/ICPC Asia Regional Contest, Shenyang Site
\item \makebox[2.6cm][l]{2016} Bronze Medal, ACM/ICPC EC-Final Contest, Shanghai Site
\item \makebox[2.6cm][l]{2015} Second Prize, China Undergraduate Mathematical Contest in Modeling
\end{innerlist}
\end{outerlist}

\blankline

\textbf{Honors and Fellowshops}
\begin{outerlist}
\item[] 
\begin{innerlist}
\item \makebox[2.6cm][l]{2023} \href{https://www.jsps.go.jp/}{Japan Society for the Promotion of Science (JSPS, 日本学術振興会) DC2 Fellowship}
\item \makebox[2.6cm][l]{2022} MSRA Collaborative Research Program Fellowship (D-CORE 2022)
\item \makebox[2.6cm][l]{2021} \href{https://spring-gx-appl.adm.s.u-tokyo.ac.jp/en/}{SPRING GX Fellowship}
\item \makebox[2.6cm][l]{2021} Excellent Graduate of Nanjing University
\item \makebox[2.6cm][l]{2020} Excellent Graduate Student of Nanjing University
\item \makebox[2.6cm][l]{2020} AAMAS 2021 Attendance Scholarship
\item \makebox[2.6cm][l]{2019} ICDM Student Travel Award
\item \makebox[2.6cm][l]{2018, 2019, 2020} First Prize, Nanjing University Academic Scholarship
\item \makebox[2.6cm][l]{2018} Second Prize, Nanjing University Yingcai Scholarship
\item \makebox[2.6cm][l]{2018} Excellent Graduate Student of Northwestern Polytechnical University
\item \makebox[2.6cm][l]{2017} First Prize, Wu Yajun Alumni Scholarship
\item \makebox[2.6cm][l]{2015, 2016} National Scholarship
\end{innerlist}
\end{outerlist}

\section{Services}
\textbf{Conference Research Services}
\begin{outerlist}
\item[]
\begin{innerlist}
\item Reviewer for ICML 2022\&2023, NeurIPS 2022\&2023, ICLR 2023\&2024, ECAI 2020, CIKM 2020, IJCAI 2021\&2022\&2023, CCML 2021
\end{innerlist}
\end{outerlist}

\textbf{Journal Research Services}
\begin{outerlist}
\item[]
\begin{innerlist}
\item Reviewer for \href{https://ieeexplore.ieee.org/xpl/RecentIssue.jsp?punumber=5962385}{IEEE Transactions on Neural Networks and Learning Systems (TNNLS)}
\end{innerlist}
\end{outerlist}

\textbf{Volunteer Work}
\begin{outerlist}
\item[]
\begin{innerlist}
\item 2021 AAMAS (Proceedings of the 20th International Conference on Autonomous Agents and Multi-Agent Systems)
\item 2018 \& 2019 MLA (China Symposium on Machine Learning and Applications)
\end{innerlist}
\end{outerlist}
% \begin{loneinnerlist}
%     \item[] \textbf{Conference Research Services} \\ Reviewer in ECAI 2020, CIKM 2020, IJCAI 2021, and CCML 2021
%     \item[] \textbf{Journal Research Services} \\ \href{https://ieeexplore.ieee.org/xpl/RecentIssue.jsp?punumber=5962385}{IEEE Transactions on Neural Networks and Learning Systems (TNNLS)}
%     \item[] \textbf{Volunteer Work} \\ 2018 \& 2019 MLA (China Symposium on Machine Learning and Applications)
% \end{loneinnerlist}

\section{Computer Skills}

\begin{loneinnerlist}
    \item[] Programming: Python, C/C++, {\sc Matlab}, Java, Shell, \LaTeX
    \item[] Operating systems: Ubuntu/Linux, Windows family
    % \item[] AI Tools: Tensorflow, Pytorch, XGBoost, Sklearn, Numpy, STL
    \item[] Languages: Chinese (Native), English (TOEFL-iBT 105/120)
\end{loneinnerlist}

% \section{Language Skills}

% \begin{loneinnerlist}
%     \item[] Chinese: Native
%     \item[] English: Professional Proficiency (TOEFL-iBT 105/120)
% \end{loneinnerlist}

\section{Hobbies}
\begin{loneinnerlist}
    \item[] Fitness, Running
\end{loneinnerlist}
%\section{Paper}
%\cite{zhang-Discriminative-Codeword-Selection,zhang-Convex-Experimental-Design,AAAI-Modeling-Dynamic,AAAI-G-optimal,Chen-CLE}
%\bibliographystyle{abbrv}
%\bibliography{ref}
\end{document}

%%%%%%%%%%%%%%%%%%%%%%%%%% End CV Document %%%%%%%%%%%%%%%%%%%%%%%%%%%%%
